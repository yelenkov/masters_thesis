% Document class and language
\documentclass{report}
\usepackage[english]{babel}
\usepackage[style=apa, sortcites=true, sorting=nyt, backend=biber]{biblatex}
\addbibresource{references/Thesis.bib}
% Style and margins
\usepackage[a4paper, top=2.5cm, bottom=2.5cm, left=3cm, right=2cm]{geometry}

% Fonts
\usepackage{fontspec}
\setsansfont{Times New Roman}
\setmainfont{Times New Roman}
\setlength{\parskip}{12pt}

% Paragraph indentation
\setlength{\parindent}{0.5cm}
\usepackage{array}

% Line spacing and table row height
\linespread{1.5}
\renewcommand{\arraystretch}{1.15}

% Text alignment
\raggedbottom

% Table column types
\newcolumntype{L}{>{\centering\arraybackslash}m{##1}}
\newcolumntype{M}{>{\centering\arraybackslash}m{##1}}
\newcolumntype{N}{>{\centering\arraybackslash}m{##1}}

% Images
\usepackage{graphicx}

% Math equations
\usepackage{amsmath}

% Hyperlinks
\usepackage[colorlinks=true, allcolors=blue]{hyperref}

% Enumerate
\usepackage{enumitem}

% code settings
\usepackage{listings}
\usepackage{xcolor}
\definecolor{codegreen}{rgb}{0,0.6,0}
\definecolor{codegray}{rgb}{0.5,0.5,0.5}
\definecolor{codeorange}{rgb}{1,0.49,0}
\definecolor{backcolour}{rgb}{0.95,0.95,0.96}

\lstdefinestyle{mystyle}{
    backgroundcolor=\color{backcolour},
    commentstyle=\color{codegray},
    keywordstyle=\color{codeorange},
    numberstyle=\tiny\color{codegray},
    stringstyle=\color{codegreen},
    basicstyle=\ttfamily\footnotesize,
    breakatwhitespace=false,
    breaklines=true,
    captionpos=b,
    keepspaces=true,
    numbers=left,
    numbersep=5pt,
    showspaces=false,
    showstringspaces=false,
    showtabs=false,
    tabsize=2,
    xleftmargin=10pt,
}

\lstset{style=mystyle}

% \lstinputlisting[language=Python, caption={Calculate PMV with pythermalcomfort.},label={lst:pmv_example}, mathescape=true, firstline=1]{snipet_1.py}

\begin{document}
\begin{titlepage}
\begin{center}
\textbf{WROCŁAW 2024}

\vspace{0cm}
\includegraphics[width=0.8\textwidth]{uewroc.jpg}

\vspace{0cm}
Program

\textbf{Business Informatics}

\vspace{0cm}
{\Large \textbf{Dawid Jeleńkowski}}

Student No. 174682

\vspace{1cm}
\textbf{MASTER’S THESIS}

\vspace{1cm}
{\huge \textbf{Leveraging Functional Programming in Scala for Efficient Data Engineering}}

\textbf{Wykorzystanie programowania funkcjonalnego w Scala do wydajnej inżynierii danych}

\vspace{2cm}
Master’s thesis written under the supervision of

{\Large \textbf{Doctor of Engineering Adam Sulich}}

\vspace{1cm}
I approve the thesis and I request for further processing

\vspace{0cm}
\line(1,0){200}

Supervisor’s signature

\vspace{0cm}
WROCŁAW 2024
\end{center}
\end{titlepage}

\begin{abstract}
Abstract in English.
\end{abstract}

\begin{abstract}
Abstract in Polish.
\end{abstract}

\tableofcontents

\chapter*{Introduction}
\addcontentsline{toc}{chapter}{Introduction}

In the introduction, I would provide background on the rise of big data and the crucial role of data engineering in managing and processing large volumes of data efficiently. Then, discuss the need for scalable and high-performance data processing frameworks like Apache Spark, and how Scala has emerged as a preferred language for data engineering due to its functional programming capabilities.

\chapter{The Functional Programming Paradigm}
%\label{chap:functional_programming}
%\addcontentsline{toc}{chapter}{\numberline{\thechapter}The Functional Programming Paradigm}

In the first chapter, I would:

\begin{itemize}
\item Explain the fundamentals of functional programming, including \emph{immutable data}, \emph{first-class functions}, \emph{avoidance of side effects}, and \emph{referential transparency}. 
\item Mention the benefits of these concepts, such as \emph{easier parallelization}, \emph{higher-order functions}, and \emph{pattern matching}, in contrast to traditional imperative paradigms.
\item Discuss how Scala flexibly combines both object-oriented and functional approaches.
\end{itemize}

\chapter{Scala for Data Engineering}

Elaborate how Scala's functional programming constructs make it well-suited for building distributed data processing systems and pipelines. Mention language features like case classes, pattern matching, higher-order functions. Discuss Scala ecosystems like Spark, Akka Streams etc. providing concurrency and distribution abstractions. 

\chapter{Proposed Approach}

Outline proposed techniques and methods to demonstrate leveraging Scala's functional capabilities for addressing data engineering challenges like:

\begin{itemize}
\item Distributed, parallel processing
\item Handling large data volumes
\item Ensuring efficiency and fault-tolerance
\item Minimizing latency
\end{itemize}

\chapter{Expected Outcomes}

Discuss the expected outcomes of the thesis in terms of performance improvements, efficiency gains, and how it advances state-of-the-art in using functional paradigms for data engineering.

\chapter{Conclusion}
Wrap up with a summary and discuss potential future work.

\appendix
\chapter{appendix}
\chapter{Bibliography}
\chapter{Other titles in this collection}

\end{document}