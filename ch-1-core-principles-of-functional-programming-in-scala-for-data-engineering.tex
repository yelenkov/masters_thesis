\chapter{Core Principles of Functional Programming in Scala for Data Engineering}
%\label{chap:functional_programing}
%\addcontentsline{toc}{chapter}{\numberline{\thechapter}The Functional programing Paradigm}

In recent years, the field of information technology has experienced exponential growth. To address the demands of managing and analyzing ever-growing data. Certain solutions were created. As organizations grapple with the challenges of big data, they require data processing systems that can be scaled, efficient, and easy to maintain. Functional programming has emerged as a potent strategy for constructing such systems, delivering advantages like immutability, modularity, and parallelism that align with data-driven demands.

The chapter explores how Scala, a multi-paradigm language that seamlessly blends object-oriented and functional programming, can be leveraged to create robust data engineering solutions. The core principles of functional programming in Scala and how they are applied specifically to data engineering tasks were examined.

The chapter begins by discussing the history and design philosophy of Scala, highlighting its ability to "grow" as a language through the incorporation of domain-specific languages. Afterward, acquire knowledge regarding the critical components that render Scala an effective tool for data engineering, including its compatibility with both functional and object-oriented approaches.

    \begin{enumerate}
        \item \textbf{Immutable data structures (IDS)} and their benefits for data consistency and concurrency;
        \item \textbf{Higher-order functions (HOFs)} for building flexible and reusable data processing pipelines;
        \item \textbf{Lazy evaluation (LE)} for efficient handling of large datasets;
        \item \textbf{Pattern matching (PM)} and \textbf{algebraic data types (ADTs)} for expressive data modeling and transformation;
        \item \textbf{Type classes (TC)} for creating generic, extensible abstractions;
        \item \textbf{Monads (M)} and functional error handling for robust data pipelines;
        \item \textbf{Parallel and distributed processing (PDP)} capabilities.
    \end{enumerate}

Throughout the chapter, code examples were provided to illustrate how these concepts can be implemented in practical data engineering scenarios.

\section{Background}

Scala stands for a \textbf{“scalable language”}. Scala was created by German computer scientist \textbf{Martin Odersky}. Odersky worked on the Java compiler and generics development at Sun Microsystems. He wanted to make a language that combines both object-oriented and functional programming styles. He wanted to make a typed language that works with the \textbf{Java Virtual Machine (JVM)}. The very first version of Scala appeared in 2003, marking the beginning of its journey. Over time, it has become a widely used language with a plethora of libraries and tools (\cite{oderskyOverviewScalaProgramming2006})\footnote[1]{\fullcite{oderskyOverviewScalaProgramming2006}}.

Scala is a language that combines ideas from \textbf{object-oriented and functional programming}. The design is inspired by several other languages. Scala uses \textbf{Java's syntax and object-oriented features}, so it's easy for Java developers to learn and works well with Java code. Scala has many things in common with Haskell and ML, like having immutable data structures, higher-order functions and pattern matching. Scala uses the ML family of languages for its type system. It has a static type system with a type interface. Scala also tries to be as short and simple as scripting languages like Python. It has features like operator overloading and implicit conversions that make DSLs (\cite{odersky.etal_2021})\footnote[2]{\fullcite{odersky.etal_2021}}.

\section{Growing a Language}

Scala's was designed with the idea of \textbf{"growing a language"}, which sets it apart from other programming languages. The language was designed to give the user only the core functionalities and the ability to build libraries upon them. Meaning, user on its own designs a new language based on Scala for solving the particular problem or issue (\cite{odersky.etal_2021})\footnotemark[2].

This concept is called \textbf{domain-specific languages} or \textbf{DSLs}, for short. In practice, these are \textbf{mini-languages} or application programming interfaces \textbf{(APIs)}. These languages can be built in an expressive and intuitive manner for the end user. Meaning that they can be read and understood by people who have nothing to do with programming but are domain experts. Scala's syntax, type system and abstraction mechanisms make the tool well suited for crafting DSLs. Features — like operator overloading, implicit conversions and higher-order functions — empower developers to craft allow developers to create APIs that feel like natural extensions of the language itself (\cite{odersky.etal_2021})\footnotemark[2].

Key factors that help a Scala evolve are its \textbf{abstraction abilities}. Scala can be used to make patterns and components that can be combined. \textbf{Traits}, for instance, allow programmers to define repeatable interfaces and implementations that can be incorporated into classes as needed. \textbf{Higher-order functions} and \textbf{type classes} help create abstractions that can be used with many types of data. These abstraction mechanisms hold significant importance in the development of libraries and frameworks that can augment the language in novel ways (\cite{odersky.etal_2021})\footnotemark[2].

Having that said, the idea of \textbf{DSLs} wouldn't make any sense without \textbf{community-driven development}. Scala contains a rich ecosystem of libraries and frameworks, of which a few are discussed in the later chapters. These libraries are often made by people who have encountered a problem and decided to use Scala to solve it. Thanks to this design, Scala can stay relevant and adapt to the changing needs of developers or the industry in general (\cite{odersky.etal_2021})\footnotemark[2].

The design of Scala ensures that \textbf{custom libraries feel like native language features}. The unified type system in Scala makes it easy to add new types — like classes, traits or objects — that can be used with other built-in types in pairs. Meaning, libraries can create new types that are as natural as Integers, Strings or Lists (\cite{ghoshDSLsAction2011})\footnotemark[3].

\section{What Makes Scala Scalable?}

Scala's smooth \textbf{integration between object-oriented and functional programming} is important for scalability. Object-oriented elements give the structure and flexibility needed for making big and complicated systems. They allow developers to create components that can be combined and expanded as the codebase grows. On the other hand, the functional aspects, support scalability. By combining these two approaches, developers can write code that's both modular and flexible to meet the changing needs of the system (\cite{odersky.etal_2021})\footnote[2]{\fullcite{odersky.etal_2021}}.

In addition, Scala's ability to use static types is significant for making it \textbf{easy to scale}. The \textbf{type system} helps find errors early, reduces errors during runtime and makes code more reliable. It supports scalable code evolution through various features that enable the creation of flexible and reusable abstractions. These abstractions can adapt to changing requirements without sacrificing type safety or requiring extensive code rewrites. It is easier to reason about as the codebase grows in size and complexity because the type system encourages developers to express their intent clearly (\cite{odersky.etal_2021})\footnotemark[2].

Lastly, Scala's high-level abstractions for data processing and concurrency are the best feature for writing scalable code. \textbf{Abstractions} — like \textbf{map}, \textbf{flatMap} and \textbf{fold} (explained later) — allow developers to express complex data transformations and computations in a concise and declarative manner. These abstractions can be used on multiple cores or machines, which makes it easy to process large datasets (\cite{odersky.etal_2021})\footnotemark[2].

In terms of \textbf{concurrency}, Scala has useful tools known as \textbf{Futures} and \textbf{Promises} that make it easier to create \textbf{asynchronous} and \textbf{non-blocking code}. Using abstractions and tools like \textbf{Akka} help developers to create \textbf{responsive} and \textbf{resilient systems} that can endure without the  need to handle low-level concurrency primitives directly (\cite{odersky.etal_2021})\footnote[2]{\fullcite{odersky.etal_2021}}.

\section{Object-Oriented Paradigm in Scala}

As previously stated, \textbf{Scala is based on the JVM}. Consequently, the code is written in a manner that is similar to that of Java. Object-oriented programming, together with Scala’s functional programing qualifies Scala as a \textbf{multi-paradigm language.} In short, as a result, \textbf{every value is an object and every operation is a method call} (\cite{joshuad.suerethScalaDepth2012})\footnote[4]{\fullcite{joshuad.suerethScalaDepth2012}}.

Object-oriented programming offers excellent \textbf{modularity} and \textbf{encapsulation}, resulting in \textbf{reusable code}. \textbf{Classes} and \textbf{objects} help organize data and behaviors into logical parts. With this approach, code becomes significantly more streamlined and organized. By putting a data into a processing logic in classes and objects, components can be made and used to build \textbf{data processing pipelines}. This modular approach makes code easier to maintain and test, which holds great significance in big data projects (\cite{ghoshDSLsAction2011})\footnotemark[3].

In addition, Scala supports \textbf{single inheritance through classes}, which allows for hierarchical relationships between data structures and processing logic. Common behaviors and attributes can be defined in base classes and specialized in derived classes, which helps eliminate redundant code and promotes code reuse (\cite{ghoshDSLsAction2011})\footnotemark[3].

\textbf{Traits} are a flexible way to combine behavior and define common interfaces. Adding functionality horizontally by mixing it into classes allows for the creation of modular and reusable data processing elements. Common data processing operations — such as \textbf{data transformation, aggregation and persistence} — can be defined across different stages of the data pipeline based on inheritance and trait composition (\cite{ghoshDSLsAction2011})\footnote[3]{\fullcite{ghoshDSLsAction2011}}.

\begin{table}[h!]
\caption{Scala's transformation of operators into method calls}
\begin{lstlisting}
val x = 1 + 2 * 3 // How it is normally written
val y = 1.+(2.*(3)) // To what the code is transformed under the hood
\end{lstlisting}
\small
\textit{Note.} This table illustrates Scala's uniform object model, where arithmetic operators are translated into method invocations on objects.
\textit{Creator.} Author's own work.
\end{table}

\section{Overview of Functional Programming Paradigm}

\textbf{Functional programming} (\textbf{FP}) is a way of making programs that uses lambda calculus. Programs are created by combining functions together. In \textbf{FP}, computation means evaluating mathematical functions without changing their state or modifying data (\cite{StenbergFunctionalAI})\footnote[5]{\fullcite{StenbergFunctionalAI}}.

Functional programming is based on lambda calculus, a formal system developed by Alonzo Church in the 1930s to study computerability. The first functional programing language, Lisp, was developed by John McCarthy in 1958. Other early influential functional languages include: APL (1966), ML (1973), Scheme (1975), Miranda (1985), Haskell (1990). In the last few years, functional programming has become more popular. Modern multi-paradigm languages like Scala, F\#, Clojure and Elixir all support functional programming. Many mainstream languages like Java, C\# and Python have also added functional features (\cite{Kunasaikaran2016ABO})\footnote[6]{\fullcite{Kunasaikaran2016ABO}}.

\section{Core Features of Functional Programing}

The following section provides an overview of \textbf{key functional programming concepts} utilized in Scala \textbf{for data engineering}. These concepts include immutable data structures, higher-order functions, lazy evaluation, pattern matching, algebraic data types, type classes, monads, error handling and parallel and distributed processing. This thesis does not attempt to provide a comprehensive explanation of functional programming or Scala. Instead, it provides a concise explanation of each concept and offers a straightforward example to demonstrate its usage. Having a solid grasp of these concepts is crucial for building efficient data pipelines. This section focuses on the core principles and how they can be used in data engineering practices.

\subsection{Immutable data structures}

\textbf{Immutable data structures} are fundamental, in functional programing. It plays a key role, in constructing data pipelines. In Scala, these types of data structures are part of the language and its standard library. An immutable data structure is one that \textbf{stays the same after it is created}. \textbf{Once its structure is set, it stays that way all the time}. The original data structure is left unchanged by any action that seems to alter the structure (\cite{zibinObjectReferenceImmutability2007})\footnote[7]{\fullcite{zibinObjectReferenceImmutability2007}}.

As the name "immutable" indicates, \textbf{data consistency} makes sure that the data stays the same throughout the data processing pipeline. By not letting changes to the data happen, there is no chance for the data to get corrupted or inconsistent. This is crucial when many processes or threads are working on the data at once (\cite{milewskiFunctionalDataStructures2013})\footnotemark[8].

Another advantage is that data structures are inherently \textbf{thread-safe}, as they cannot be modified by multiple threads simultaneously. This eliminates the necessity for synchronization mechanisms. In addition, it helps to reduce the likelihood of race conditions and other bugs related to concurrency. Data can be easily shared and accessed by multiple threads without the need for locks or other synchronization primitives (\cite{milewskiFunctionalDataStructures2013})\footnotemark[8].

Another notable benefit is \textbf{fault tolerance}. It is the ability to handle errors and continue functioning effectively. Regardless of any circumstances, the data remains consistent. The processing can resume from a recognized point. This feature is incredibly valuable for systems that experience frequent failures and require the ability to recover from setbacks in order to maintain accurate and consistent data (\cite{milewskiFunctionalDataStructures2013})\footnote[8]{\fullcite{milewskiFunctionalDataStructures2013}}.

Scala has many types of immutable data — like \textbf{List}, \textbf{Vector}, \textbf{Map} and \textbf{Set} — in its standard library. These\textbf{ data structures} (\textbf{DS}) were created to maximize efficiency. Tailored to practical scenarios. For example, List is a \textbf{linked-list} implementation that can be used for recursive algorithms and pattern matching. On the other hand, the \textbf{Vector} \textbf{DS} is based on trees, which provide impressive random access and updates, making it perfect for indexed data processing  (\cite{scalaVector})\footnote[9]{\fullcite{scalaVector}}.

\begin{table}[h!]
\caption{Immutability and transforming lists}
\begin{lstlisting}
val data = List(1, 2, 3, 4, 5)
val updatedData = data.map(_ * 2)
println(data)       // Output: List(1, 2, 3, 4, 5)
println(updatedData) // Output: List(2, 4, 6, 8, 10)
\end{lstlisting}
\small
\textit{Note.} In this example, the \textbf{data} list is an immutable data structure. The \textbf{map} operation creates a new list, \textbf{updatedData} by applying the transformation function to each element of \textbf{data}. The original \textbf{data} list remains unchanged, ensuring data consistency and thread safety.
\textit{Creator.} Author's own work.
\end{table}

In addition, when data cannot be changed, it allows for easy sharing  between different parts of the data processing pipeline. Since the information stays the same, it can be shared and used again without worrying about copying or synchronizing. It especially shines when dealing with large datasets by improving performance (\cite{tomeDataEngineeringScala2024})\footnote[10]{\fullcite{tomeDataEngineeringScala2024}}.

\begin{table}[H]
\caption{Partitioning a vector based on a predicate}
\begin{lstlisting}
val data = Vector(1, 2, 3, 4, 5)
val evenData = data.filter(_ % 2 == 0)
val oddData = data.filter(_ % 2 != 0)
println(evenData) // Output: Vector(2, 4)
println(oddData)  // Output: Vector(1, 3, 5)
\end{lstlisting}
\small
\textit{Note.} In this example, the \textbf{data} vector is shared between the \textbf{evenData} and \textbf{oddData} computations. Since \textbf{data} is immutable, it can be safely accessed and filtered by both computations without the risk of data races or inconsistencies.
\textit{Creator.} Author's own work.
\end{table}

The fact that data is immutable, results in the ability to create and use \textbf{lazy evaluation} (\textbf{LE}) and optimize data processing pipelines. \textbf{LE} lets delay the computation of intermediate results until they are needed, which results in reduced memory overhead and improved overall performance. Scala's data structures — such as \textbf{Stream} and \textbf{Iterator} — provide \textbf{LE} capabilities out of the box, which makes it easy to create memory-friendly data pipelines (\cite{tomeDataEngineeringScala2024})\footnote[10]{\fullcite{tomeDataEngineeringScala2024}}.

\begin{table}[h!]
\caption{Example of lazy evaluation and infinite sequences}
\begin{lstlisting}
val data = Stream.from(1)
val evenData = data.filter(_ % 2 == 0).take(5)
println(evenData.toList) // Output: List(2, 4, 6, 8, 10)
\end{lstlisting}
\small
\textit{Note.} In this example, the \textbf{data} stream is an infinite sequence of integers starting from 1. The \textbf{filter} operation lazily filters the even numbers from the stream, and the \textbf{take} operation limits the result to the first 5 even numbers. The computation is performed lazily, generating only the required elements on-demand, thus avoiding the need to materialize the entire infinite sequence in memory.
\textit{Creator.} Author's own work.
\end{table}

\subsection{Higher-Order Functions}

\textbf{Higher-order functions} (\textbf{HOFs}) are yet another killer feature; basically, they help make data processing systems that can be easily changed and reused. Through their utilization, the code can be fine-tuned to be faster and easier to maintain. These functions typically used for data transformation, grouping and analysis (\cite{michael.etal_2023})\footnote[11]{\fullcite{michael.etal_2023}}.

The \textbf{HOF} can be defined as a type of function that takes one or more functions as inputs, gives back a function as an output, or does both. In more complex terms, this feature of treating \textbf{functions as first-class citizens} permits the development of adaptable data processing tasks. These functions help change behavioral parameters and combine functions. (\cite{michael.etal_2023})\footnotemark[11].

In the case of examples, the Scala standard library provides a rich set of tools that are commonly used in data engineering tasks. These functions operate on collections and enable concise and expressive data manipulation. Below are the most popular ones (\cite{michael.etal_2023})\footnotemark[11].

\begin{table}[h!]
\caption{map}
\begin{lstlisting}
val data = List(1, 2, 3, 4, 5)
val squaredData = data.map(x => x * x)
println(squaredData) // Output: List(1, 4, 9, 16, 25)
\end{lstlisting}
\small
\textit{Note.} The \textbf{map} function applies a given function to each element of a collection and returns a new collection with the transformed elements. It is used for element-wise transformations and is a fundamental building block of data processing pipelines.
\textit{Creator.} Author's own work.
\end{table}

\begin{table}[h!]
\caption{flatMap}
\begin{lstlisting}
val data = List("hello world", "functional programming", "data engineering")
val words = data.flatMap(_.split(" "))
println(words) // Output: List("hello", "world", "functional", "programming", "data", "engineering")
\end{lstlisting}
\small
\textit{Note.} The \textbf{flatMap} function applies a given function to each element of a collection and flattens the resulting collections into a single collection. It is used for transformations that produce zero or more elements per input element, and is particularly useful for data flattening and joining operations.
\end{table}

\begin{table}[h!]
\caption{filter}
\begin{lstlisting}
val data = List(1, 2, 3, 4, 5)
val evenData = data.filter(_ % 2 == 0)
println(evenData) // Output: List(2, 4)
\end{lstlisting}
\small
\textit{Note.} The \textbf{filter} function selects elements from a collection based on a given predicate function. It is used for data filtering and selection operations, allowing for the creation of subsets of data that satisfy specific criteria.
\textit{Creator.} Author's own work.
\end{table}

\begin{table}[h!]
\caption{reduce}
\begin{lstlisting}
val data = List(1, 2, 3, 4, 5)
val sum = data.reduce(_ + _)
println(sum) // Output: 15
\end{lstlisting}
\small
\textit{Note.} The \textbf{reduce} function combines the elements of a collection using a binary operator function. It is used for data aggregation and summarization tasks, such as computing sums, products, or custom aggregations.
\textit{Creator.} Author's own work.
\end{table}

As shown in the 1.9 table, \textbf{HOFs} enable data processing operations that can be reused regardless of the data type. With the ability to establish the internal structure of systems using functions, one can create data processing pipelines that are adaptable to various data types and requirements (\cite{michael.etal_2023})\footnotemark[11].

\textbf{HOFs} also enable the creation of \textbf{domain-specific languages} (\textbf{DSLs}) and fluent \textbf{APIs} for data processing. By defining a set of \textbf{HOFs}, data engineers can make code that looks like the problem domain (\cite{michael.etal_2023})\footnote[11]{\fullcite{michael.etal_2023}}.

\begin{table}[h!]
\caption{Generic data processing pipeline with higher-order functions}
\begin{lstlisting}
def processData[A, B](data: List[A],
    transformFn: A => B, filterFn: B => Boolean, aggregateFn: (B, B) => B): B = {
  data.map(transformFn)
      .filter(filterFn)
      .reduce(aggregateFn)}
val data = List(1, 2, 3, 4, 5)
val sum = processData(data, x => x * x, x => x > 10, _ + _)
println(sum) // Output: 41
\end{lstlisting}
\small
\textit{Note.} In this example, the \textbf{processData} function is a higher-order function that takes three function parameters: \textbf{transformFn} for data transformation, \textbf{filterFn} for data filtering, and \textbf{aggregateFn} for data aggregation. By providing different functions as arguments, the \textbf{processData} function can be easily customized for different data processing scenarios, promoting code reuse and modularity.
\textit{Creator.} Author's own work.
\end{table}

\begin{table}[h!]
\caption{Building a data processing DSL}
\begin{lstlisting}
case class DataPipeline[A](data: List[A]) {
  def map[B](f: A => B): DataPipeline[B] = DataPipeline(data.map(f))
  def filter(f: A => Boolean): DataPipeline[A] = DataPipeline(data.filter(f))
  def reduce[B >: A](f: (B, B) => B): B = data.reduce(f)}
val pipeline = DataPipeline(List(1, 2, 3, 4, 5))
                .map(x => x * x)
                .filter(x => x > 10)
                .reduce(_ + _)

println(pipeline) // Output: 41
\end{lstlisting}
\small
\textit{Note.} In this example, the \textbf{DataPipeline} class defines a fluent API for data processing using higher-order functions. The \textbf{map}, \textbf{filter}, and \textbf{reduce} methods enable the creation of expressive and readable data processing pipelines that can be easily composed and chained together.
\textit{Creator.} Author's own work.
\end{table}

\subsection{Lazy Evaluation}

\textbf{Lazy evaluation} \textbf{(LE)} is a technique where the evaluation of an expression is delayed until its value is actually needed  — look at table 1.11. It makes big amounts of information work faster, avoids unnecessary work and let make endless data structures (\cite{scalaLazy})\footnote[12]{\fullcite{scalaLazy}}.

In Scala, \textbf{LE} is possible by using \textbf{Lazy Values} (\textbf{LV}), \textbf{Lazy Collections} (\textbf{LC}) and \textbf{lazy data processing}. In terms of \textbf{LV}, they  are only evaluated when they are accessed for the first time. Later accesses to a \textbf{LV} reuse the result that was previously computed (\cite{michael.etal_2023})\footnote[11]{\fullcite{michael.etal_2023}}.

\begin{table}[H]
\caption{Lazy evaluation}
\begin{lstlisting}
lazy val data = {
  println("Initializing data...")
  List(1, 2, 3, 4, 5)}
println("Before accessing data")
val result = data.map(_ * 2)
println("After accessing data")
\end{lstlisting}
\small
\textit{Note.} In this example, the \textbf{data} value is defined as a lazy value. The initialization code inside the block is not executed until \textbf{data} is accessed for the first time. This allows for the deferred execution of potentially expensive computations and enables the creation of data processing pipelines that can handle large datasets efficiently.
\textit{Creator.} Author's own work.
\end{table}

In the case of \textbf{LC}, — such as \textbf{Stream} and \textbf{Iterator} — a lazy and incremental approach to processing data is provided by these collections, which represent potentially infinite sequences of elements. A \textbf{LCs} elements are computed on demand, which makes it easy to process large or infinite datasets (\cite{hughesWhyFunctionalProgramming1990})\footnote[13]{\fullcite{hughesWhyFunctionalProgramming1990}}.

\begin{table}[h!]
\caption{Infinite fibonacci stream}
\begin{lstlisting}
def fibonacci: Stream[Int] = 0 #:: 1 #:: fibonacci.zip(fibonacci.tail).map(t => t._1 + t._2)
val fibs = fibonacci.take(10).toList
println(fibs) // Output: List(0, 1, 1, 2, 3, 5, 8, 13, 21, 34)
\end{lstlisting}
\small
\textit{Note.} In this example, the \textbf{fibonacci} function defines an infinite stream of Fibonacci numbers using lazy evaluation. The \textbf{\#::} operator is used to construct the stream lazily, and the \textbf{zip} and \textbf{map} operations are used to generate the next Fibonacci number based on the previous two. The \textbf{take} operation limits the evaluation to the first 10 elements, avoiding the need to compute the entire infinite sequence.
\textit{Creator.} Author's own work.
\end{table}

A further perk of \textbf{LE} is its applicability in building massive data processing systems that can manage petabytes or terabytes of data. Additionally, \textbf{LE} can be implemented to incrementally process data, with an emphasis on the most critical computations. Furthermore, \textbf{LE} can be employed to process data incrementally, focusing solely on the essential computations (\cite{chenE3ElasticExecution2011})\footnote[14]{\fullcite{chenE3ElasticExecution2011}}.

In the case of Scala's functional programing libraries, — such as \textbf{Apache Spark} and \textbf{Akka Stream}; table 1.12 — they use \textbf{LE} a lot to process data in many places at once. In \textbf{Apache Spark}, \textbf{LE} is used to build a \textbf{directed acyclic graph} (\textbf{DAG}) — as can be seen in the 1.13 table — of transformations and actions that are optimized and executed in a distributed way across a cluster of machines. \textbf{Spark} can use \textbf{LE} to make the execution plan work better, move data less often and improve performance overall (\cite{michael.etal_2023})\footnote[11]{\fullcite{michael.etal_2023}}.

\begin{table}[H]
\caption{Direct acrylic graph (DAG)}
\begin{lstlisting}
val data = spark.textFile("hdfs://path/to/data.txt")
val words = data.flatMap(_.split(" "))
val wordCounts = words.map((_, 1)).reduceByKey(_ + _)
wordCounts.saveAsTextFile("hdfs://path/to/output")
\end{lstlisting}
\small
\textit{Note.} In this example, the \textbf{data} RDD (Resilient Distributed Dataset) represents a large text file stored in HDFS. The \textbf{flatMap}, \textbf{map}, and \textbf{reduceByKey} operations are lazily evaluated, building a DAG of transformations. The actual computation is triggered only when the \textbf{saveAsTextFile} action is called, allowing Spark to optimize the execution plan and distribute the processing across the cluster.
\textit{Creator.} Author's own work.
\end{table}

By utilizing lazily defined data transformations and computations, data engineers can construct reusable and adaptable components that can be easily combined and adapted to diverse data processing scenarios — look at table 1.14. This strategy encourages code reuse, improves upkeep and aids in the development of intricate data processing workflows (\cite{michael.etal_2023})\footnote[14]{\fullcite{michael.etal_2023}}.

\begin{table}[h!]
\caption{Data processing workflow}
\begin{lstlisting}
def loadData(path: String): Stream[String] = {
  val source = io.Source.fromFile(path)
  val lines = source.getLines().toStream
  source.close()
  lines}
  
def processData(data: Stream[String]): Stream[(String, Int)] = {
  data.flatMap(_.split(" "))
      .map((_, 1))
      .groupBy(_._1)
      .mapValues(_.map(_._2).sum)
      .toStream
}
val data = loadData("path/to/data.txt")
val processed = processData(data)

processed.take(10).foreach(println)
\end{lstlisting}
\small
\textit{Note.} In this example, the \textbf{loadData} function lazily loads data from a file and returns a stream of lines. The \textbf{processData} function defines a series of lazy transformations on the input stream, including splitting lines into words, counting word occurrences, and grouping by word. The \textbf{take} operation triggers the evaluation of the first 10 elements, and the \textbf{foreach} operation prints the results. This modular and composable approach allows for the creation of reusable data processing components that can be easily combined and adapted to different datasets and requirements.
\textit{Creator.} Author's own work.
\end{table}

\subsection{Pattern Matching and Algebraic Data Types (ADTs)}

\textbf{Pattern matching} and \textbf{algebraic data types} — for shorts,  \textbf{PM} and \textbf{ADTs} — provide an expressive way to model, process and retrieve information from complex data structures. \textbf{PM} makes code clear and easy to understand, and \textbf{ADTs} make data structures that are expressive and type-safe — look at table 1.15 (\cite{michael.etal_2023})\footnotemark[11].

Despite the complex name, \textbf{ADTs} are just a way to define complex data structures using a combination of product types — such as \textbf{tuples} or \textbf{case classes} — and summarize types — like \textbf{sealed traits} or \textbf{enums} —. Product types are data with multiple fields, while summarize types are data with only one field. What makes \textbf{ADTs} useful is that they allow for the modeling of hierarchical and recursive data structures (\cite{michael.etal_2023})\footnotemark[11].

They are typically defined using \textbf{sealed traits} and \textbf{case classes}. The first ones define a set of possible variants that covers all cases and prevents the addition of new variants outside the set. Case classes, on the other hand, represent the individual variants of the \textbf{ADT} and provide a convenient way to define and manipulate data (\cite{michael.etal_2023})\footnote[11]{\fullcite{michael.etal_2023}}.

\begin{table}[h!]
\caption{Data model using ADTs}
\begin{lstlisting}
sealed trait Shape
case class Circle(radius: Double) extends Shape
case class Rectangle(width: Double, height: Double) extends Shape
case class Triangle(base: Double, height: Double) extends Shape
\end{lstlisting}
\small
\textit{Note.} In this example, the \textbf{Shape} ADT represents different geometric shapes. It is defined as a sealed trait, and the individual shapes (Circle, Rectangle, Triangle) are defined as case classes that extend the \textbf{Shape} trait. This allows for the creation of a type-safe and expressive data model for representing shapes.
\textit{Creator.} Author's own work.
\end{table}

As mentioned before, the \textbf{PM} is a feature in Scala that allows for the concise and expressive processing of \textbf{ADTs} — look at table 1.16 —. Basically, it enables the deconstruction and extraction of data based on its structure and variant (\cite{michael.etal_2023})\footnotemark[11].

\begin{table}[h!]
\caption{Pattern matching}
\begin{lstlisting}
def area(shape: Shape): Double = shape match {
  case Circle(radius) => math.Pi * radius * radius
  case Rectangle(width, height) => width * height
  case Triangle(base, height) => 0.5 * base * height}
\end{lstlisting}
\small
\textit{Note.} In this example, the \textbf{area} function uses pattern matching to calculate the area of different shapes. The \textbf{match} expression checks the type of the \textbf{shape} parameter and extracts the relevant fields based on the corresponding case class. This allows for concise and expressive code that handles each shape variant separately.
\textit{Creator.} Author's own work.
\end{table}

\textbf{PM} can also be used to extract and reorganize data. It is done by using patterns and conditions to get data that needs to be filtered, transformed or grouped based on certain criteria  — look at table 1.17 (\cite{michael.etal_2023})\footnotemark[11].

\begin{table}[h!]
\caption{Data processing and aggregation}
\begin{lstlisting}
def processData(data: List[Shape]): Double = data match {
  case Nil => 0.0
  case Circle(radius) :: rest => radius * radius + processData(rest)
  case Rectangle(width, height) :: rest => width * height + processData(rest)
  case Triangle(base, height) :: rest => 0.5 * base * height + processData(rest)}
\end{lstlisting}
\small
\textit{Note.} In this example, the \textbf{processData} function uses pattern matching to process a list of shapes. It recursively matches on the head of the list and extracts the relevant fields based on the shape variant. The function calculates a value based on the shape and recursively processes the rest of the list. This demonstrates how pattern matching can be used for data processing and aggregation.
\textit{Creator.} Author's own work.
\end{table}

For example, when a data engineer has to work with complex and hierarchical data structures. He can create type-safe data models that can be easily processed and transformed — look at table 1.18 (\cite{michael.etal_2023})\footnote[11]{\fullcite{michael.etal_2023}}.

\begin{table}[h!]
\caption{Pattern matching and ADTs}
\begin{lstlisting}
case class User(id: Int, name: String, age: Int)
case class Transaction(userId: Int, amount: Double, timestamp: Long)

def processTransactions(transactions: List[Transaction]): Map[Int, Double] = transactions match {
  case Nil => Map.empty[Int, Double]
  case Transaction(userId, amount, _) :: rest =>
    val userTotal = processTransactions(rest).getOrElse(userId, 0.0) + amount
    processTransactions(rest) + (userId -> userTotal)
}

val users = List(User(1, "Alice", 25), User(2, "Bob", 30), User(3, "Charlie", 35))

val transactions = List(
  Transaction(1, 100.0, System.currentTimeMillis()),
  Transaction(2, 50.0, System.currentTimeMillis()),
  Transaction(1, 75.0, System.currentTimeMillis()),
  Transaction(3, 120.0, System.currentTimeMillis())

val userTotals = processTransactions(transactions)
val userSummary = users.map(user => (user, userTotals.getOrElse(user.id, 0.0)))
\end{lstlisting}
\small
\textit{Note.} In this example, ADTs is defined for \textbf{User} and \textbf{Transaction}. The \textbf{processTransactions} function uses pattern matching to process a list of transactions and calculate the total amount for each user. The \textbf{userSummary} combines the user information with their transaction totals using pattern matching and the \textbf{map} operation. This demonstrates how pattern matching and ADTs can be used together to process and analyze complex data structures in a data engineering context.
\textit{Creator.} Author's own work.
\end{table}



\subsection{Type Classes}

In functional programing, \textbf{type classes} — for short, \textbf{TC} — are an idea that allows for ad-hoc polymorphism and offers a way to define generic behavior for types without altering their primary definitions. These classes are crucial as they facilitate the development of reusable and composable abstractions that're applicable, to various data types and structures (\cite{odersky.etal_2021})\footnote[2]{\fullcite{odersky.etal_2021}}. \textbf{TCs} are implemented using implicit parameters and implicit definitions — look at table 1.19 —; a \textbf{TC} is defined as a trait that declares a set of operations or behaviors that can be implemented for different types. These operations are defined as abstract methods within the trait. Types that want to belong to the \textbf{TCs} provide implicit implementations of these methods (\cite{odersky.etal_2021})\footnotemark[2].

\begin{table}[h!]
\caption{Implicit implementations}
\begin{lstlisting}
trait Semigroup[A] {
  def combine(x: A, y: A): A}
object Semigroup {
  implicit val intSemigroup: Semigroup[Int] = new Semigroup[Int] {
    def combine(x: Int, y: Int): Int = x + y}
  implicit val stringSemigroup: Semigroup[String] = new Semigroup[String] {
    def combine(x: String, y: String): String = x + y}}
\end{lstlisting}
\small
\textit{Note.} In this example, a \textbf{Semigroup} type class is defined that represents types with an associative binary operation \textbf{combine}. Implicit implementations of \textbf{Semigroup} for \textbf{Int} and \textbf{String} types are provided, defining the \textbf{combine} operation as addition and concatenation, respectively.
\textit{Creator.} Author's own work.
\end{table}

TCs help create generic functions — look at table 1.20 —; These are the functions that can be used with any type that has an implicit instance of the \textbf{TC} needed. In other words, it makes it possible to make abstractions that can be used on different data types without changing their original definitions (\cite{odersky.etal_2021})\footnotemark[2].

\begin{table}[h!]
\caption{Generic function}
\begin{lstlisting}
def combineAll[A](values: List[A])(implicit semigroup: Semigroup[A]): A = values.reduce(semigroup.combine)
val numbers = List(1, 2, 3, 4, 5)
val strings = List("Hello", ", ", "world", "!")
println(combineAll(numbers)) // Output: 15
println(combineAll(strings)) // Output: "Hello, world!"
\end{lstlisting}
\small
\textit{Note.} In this example, the \textbf{combineAll} function takes a list of values of type \textbf{A} and an implicit \textbf{Semigroup} instance for type \textbf{A}. It uses the \textbf{combine} operation provided by the \textbf{Semigroup} to reduce the list of values into a single value. The function can be called with any type that has an implicit \textbf{Semigroup} instance, such as \textbf{Int} and \textbf{String}, without requiring any modifications to the function itself.
\textit{Creator.} Author's own work.
\end{table}

TCs are handy when there is a need to define common behaviors or operations for different data types — despite lots of code table 1.21 explains it pretty well —; They can be used to make data processing abstractions that can be used in a wide range of data structures (\cite{odersky.etal_2021})\footnotemark[2].

\begin{table}[h!]
\caption{Implicit instance}
\begin{lstlisting}
trait Encoder[A] {def encode(value: A): String}
object Encoder {
  implicit val intEncoder: Encoder[Int] = new Encoder[Int] {
    def encode(value: Int): String = value.toString}
  implicit val stringEncoder: Encoder[String] = new Encoder[String] {
    def encode(value: String): String = value}
  implicit def listEncoder[A](implicit encoder: Encoder[A]): Encoder[List[A]] = new Encoder[List[A]] {
    def encode(values: List[A]): String = values.map(encoder.encode).mkString("[", ",", "]")}}
def processData[A](data: List[A])(implicit encoder: Encoder[A]): String =
  data.map(encoder.encode).mkString(",") 
val numbers = List(1, 2, 3, 4, 5)
val strings = List("apple", "banana", "orange")
val nested = List(List(1, 2), List(3, 4), List(5, 6))
println(processData(numbers)) // Output: "1,2,3,4,5"
println(processData(strings)) // Output: "apple,banana,orange"
println(processData(nested)) // Output: "[1,2],[3,4],[5,6]"
\end{lstlisting}
\small
\textit{Note.} In this example, an \textbf{Encoder} type class is defined, it provides an \textbf{encode} operation to convert values of type \textbf{A} to a string representation. Implicit instances of \textbf{Encoder} for \textbf{Int} and \textbf{String} types are provided. Also, an implicit \textbf{listEncoder} is defined, it can encode a list of values of type \textbf{A}, given an implicit \textbf{Encoder} instance for type \textbf{A}.
The \textbf{processData} function takes a list of values of type \textbf{A} and an implicit \textbf{Encoder} instance for type \textbf{A}. It uses the \textbf{encode} operation to convert each value to its string representation and concatenates them into a single string. The function can be called with any type that has an implicit \textbf{Encoder} instance, including nested lists, demonstrating the composability and reusability of type classes.
\textit{Creator.} Author's own work.
\end{table}

In terms of Scala libraries and \textbf{TCs} the Cats and Scalaz are the \textbf{TC} are also commonly used in Scala's functional programing libraries, such as Cats and Scalaz, which provide a wide range of type classes for various algebraic structures and data processing operations. These libraries leverage type classes to define generic and composable abstractions for data processing — such as Functor, Monad and Traverse — which can be applied to different data types and structures (\cite{odersky.etal_2021})\footnote[2]{\fullcite{odersky.etal_2021}}.

\subsection{Monads and error handling}

One of the hardest to understand of common concepts in Functional programming are \textbf{Monads} (\textbf{M}). In essence, they can be used to make and chain data processing operations, handle errors or validate data. Basically, a \textbf{Monad} is a design pattern that allows for the composition of computations in a way that is both expressive and type-safe. In simpler terms, it is a structure that wraps a value (or computation) and provides a standardized way to chain operations on that value
 (\cite{wadler1992monads})\footnote[15]{\fullcite{wadler1992monads}}.

The two fundamental operations are \textbf{flatMap} — also known as \textbf{bind} — and \textbf{pure} — also known as \textbf{unit} or \textbf{return} —. These operations follow certain rules called the monadic laws; basically, they make sure that they behave the same way (\cite{wadler1992monads})\footnote[15]{\fullcite{wadler1992monads}}.

In Scala, \textbf{Monads} are implemented using the \textbf{flatMap}, \textbf{map} and other \textbf{pure methods}. \textbf{A function can be called pure when it produces the same output for a given input, no matter what}. In this context, the flatMap binds computations that may produce a \textbf{Monad} instance, while the map method applies a function to the value inside the \textbf{M}. A value is inserted into the \textbf{M} context by the pure approach (\cite{joshuad.suerethScalaDepth2012})\footnote[4]{\fullcite{joshuad.suerethScalaDepth2012}}.

In the Scala's standard library there are \textbf{Monad} implementations for the most common types — such as \textbf{Option}, \textbf{Either}, \textbf{Try} and \textbf{Future} —, each serving different purposes in error handling and data processing. The most popular scenario for using \textbf{Ms} is data validation and error handling. \textbf{Ms} — like \textbf{Option} and \textbf{Either} — can be used to represent errors or missing values in a data pipeline. \textbf{Option} is used to represent a value that may or may not be present, while \textbf{Either} represents a value that can be either a success or a failure. In the result, the data pipeline remains in a consistent state and avoids unexpected failures (\cite{joshuad.suerethScalaDepth2012})\footnotemark[4].

For example, let's think about a data pipeline that involves a typical ETL scenario. Each step in the process can go wrong — like there is a missing file, record or a bug in transformation —; To solve these problems, \textbf{Either} can be used for centralized error handling and reporting by capturing these errors through the pipeline (\cite{joshuad.suerethScalaDepth2012})\footnotemark[4].

\begin{table}[H]
\caption{Monads and error handling}
\begin{lstlisting}
import scala.io.Source
def readFile(filename: String): Either[String, List[String]] =
  try {val lines = Source.fromFile(filename).getLines().toList
    Right(lines)} catch {case e: Exception => Left(s"Error reading file: ${e.getMessage}")}
def parseData(lines: List[String]): Either[String, List[Int]] =
  try {val parsed = lines.map(_.toInt)
    Right(parsed)} catch {case e: NumberFormatException => Left(s"Error parsing data: ${e.getMessage}")}
def processData(data: List[Int]): Either[String, Int] =
  try {val result = // Perform data processing
    Right(result)} catch {case e: Exception => Left(s"Error processing data: ${e.getMessage}")}
val pipeline = for {
  lines <- readFile("data.txt")
  parsed <- parseData(lines)
  result <- processData(parsed)
} yield result
pipeline match {
  case Right(result) => println(s"Data processing successful: $result")
  case Left(error)   => println(s"Data processing failed: $error")}
\end{lstlisting}
\small
\textit{Note.} In this example, the \textbf{readFile}, \textbf{parseData}, and \textbf{processData} functions return an \textbf{Either} monad, representing either a successful result or an error message. The \textbf{pipeline} is constructed using a for-comprehension, which chains the operations together using the \textbf{flatMap} and \textbf{map} methods of \textbf{Either}. If any step in the pipeline fails, the error is propagated, and the final result is a \textbf{Left} containing the error message. If all steps succeed, the final result is a \textbf{Right} containing the processed data.
\textit{Creator.} Author's own work.
\end{table}



\subsection{Parallel and Distributed Processing}

\textbf{Parallel and distributed processing} — \textbf{PDP} — in essence, responsible for the efficient processing of large volumes of data. Scala, with its libraries and frameworks, is the killer language for \textbf{PDP}. Immutable data structures make it safe to share and process data without the risk of race conditions or inconsistencies. Pure functions are inherently parallelizable and can be safely executed independently on multiple nodes in a distributed system (\cite{tomeDataEngineeringScala2024})\footnote[10]{\fullcite{tomeDataEngineeringScala2024}}.

The Parallel Collections library provides a base for parallel processing in Scala. It enables the parallel execution of operations on collections. They can be used to parallelize data processing tasks — such as mapping, filtering and aggregation, table 1.23 — across multiple CPU cores. In result, higher performance, especially on big data, is achieved. Parallel processing can accelerate a core work, but it's better to use more than one core to process a high volume of data (\cite{tomeDataEngineeringScala2024})\footnotemark[10].

\begin{table}[h!]
\caption{Parallel vector}
\begin{lstlisting}
import scala.collection.parallel.immutable.ParVector
val data: ParVector[Int] = ParVector.range(1, 1000000)
val processed = data.map(_ * 2).filter(_ % 3 == 0).reduce(_ + _)
\end{lstlisting}
\small
\textit{Note.} In this example, the data is a parallel vector containing a large number of integers. The \textbf{map}, \textbf{filter}, and \textbf{reduce} operations are executed in parallel, utilizing multiple CPU cores to process the data efficiently.
\textit{Creator.} Author's own work.
\end{table}

Scala has a rich ecosystem of libraries and frameworks that facilitate distributed data processing, such as Apache Spark and Akka. The first one provides high-level APIs for processing large datasets across clusters of machines. Spark's \textbf{Resilient Distributed Datasets}  \textbf{(RDD)} and \textbf{DataFrames} \textbf{(DF)} allow for the distributed storage and processing of structured and unstructured data — table 1.24 (\cite{tomeDataEngineeringScala2024})\footnotemark[10].

\begin{table}[h!]
\caption{Apache Spark data processing}
\begin{lstlisting}
import org.apache.spark.sql.SparkSession
val spark = SparkSession.builder().appName("DataProcessing").getOrCreate()
val data = spark.read.textFile("hdfs://path/to/data.txt")
val processed = data.flatMap(_.split(" ")).map((_, 1)).reduceByKey(_ + _).collect()
\end{lstlisting}
\small
\textit{Note.} In this example, \textbf{Spark} is used to process a large text file stored in HDFS. The \textbf{flatMap} operation splits each line into words, the \textbf{map} operation converts each word into a tuple of \textbf{(word, 1)}, and the \textbf{reduceByKey} operation counts the occurrences of each word. The \textbf{collect} operation brings the results back to the driver program.
\textit{Creator.} Author's own work.
\end{table}

The second framework is \textbf{Akka}. It has an actor model that provides a high-level abstraction for \textbf{concurrent and distributed computing}, which means that a \textbf{scalable} and \textbf{fault-tolerant} system can be made of them. Basically, actors are lightweight, independent entities that communicate through \textbf{asynchronous message passing}, enabling the distribution of data processing tasks across a cluster of nodes (\cite{tomeDataEngineeringScala2024})\footnotemark[10].

\begin{table}[h!]
\caption{Akka actors}
\begin{lstlisting}
import akka.actor.{Actor, ActorSystem, Props}
case class ProcessData(data: List[Int])
case class ProcessedData(result: Int)
class DataProcessor extends Actor {
  def receive = {case ProcessData(data) => val result = data.map(_ * 2).filter(_ % 3 == 0).sum
      sender() ! ProcessedData(result)}}
val system = ActorSystem("DataProcessingSystem")
val processor = system.actorOf(Props[DataProcessor], "processor")
val data = List(1, 2, 3, 4, 5)
processor ! ProcessData(data)
\end{lstlisting}
\small
\textit{Note.} In this example, an Akka actor system is created, and a \textbf{DataProcessor} actor is defined. The actor receives a \textbf{ProcessData} message containing a list of integers, processes the data, and sends back a \textbf{ProcessedData} message with the result. The actor can be distributed across multiple nodes in a cluster, allowing for the parallel and distributed processing of data.
\textit{Creator.} Author's own work.
\end{table}