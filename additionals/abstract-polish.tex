\begin{abstract}
W niniejszej rozprawie zbadano potencjał programowania funkcjonalnego w języku Scala w zakresie wydajnej inżynierii danych. W miarę jak organizacje zmagają się z coraz większymi ilościami danych, potrzeba skalowalnych, łatwych w utrzymaniu i wydajnych systemów przetwarzania danych stała się nadrzędna. Scala, wieloparadygmatowy język, który płynnie łączy koncepcje programowania obiektowego i funkcjonalnego, jest badany jako potężne narzędzie do sprostania tym wyzwaniom. Badane są podstawowe zasady programowania funkcyjnego w Scali, w tym immutable data structures, higher-order functions, lazy evaluation, pattern matching, algebraic data types, type classes i monady. Koncepcje te są demonstrowane w celu zapewnienia solidnych podstaw do budowania wydajnych strumieni danych. Analizowany jest cały cykl pracy związany z inżynierią danych, od pozyskiwania danych po ich udostępnianie, z naciskiem na to, w jaki sposób paradygmaty programowania funkcjonalnego mogą być stosowane na każdym etapie. Badane są techniki równoległego i rozproszonego przetwarzania danych, prezentując możliwości Scali w zakresie obsługi operacji na danych na dużą skalę. Szczególną uwagę poświęcono przetwarzaniu strumieniowemu, odzwierciedlając jego rosnące znaczenie w nowoczesnych architekturach danych. Oceniono popularne biblioteki przetwarzania strumieniowego w Scali i podkreślono ich mocne strony w analizie danych w czasie rzeczywistym. Omówiono krytyczne aspekty testowania, wdrażania i monitorowania strumieni danych. Przeanalizowano różne metodologie testowania dostosowane do Scali, podkreślając znaczenie kompleksowego testowania w zapewnianiu niezawodnych i wydajnych systemów przetwarzania danych. Omówiono strategie wdrażania, w tym wdrażanie w chmurze, wraz z najlepszymi praktykami monitorowania i optymalizacji potoków danych w środowiskach produkcyjnych. Dzięki tej wszechstronnej eksploracji wykazano, że programowanie funkcjonalne w Scali oferuje potężny zestaw narzędzi do radzenia sobie ze złożonością nowoczesnej inżynierii danych.
\end{abstract}