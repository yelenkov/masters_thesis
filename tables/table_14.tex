\begin{table}[h!]
\caption{Data processing workflow}
\begin{lstlisting}
def loadData(path: String): Stream[String] = {
  val source = io.Source.fromFile(path)
  val lines = source.getLines().toStream
  source.close()
  lines}
  
def processData(data: Stream[String]): Stream[(String, Int)] = {
  data.flatMap(_.split(" "))
      .map((_, 1))
      .groupBy(_._1)
      .mapValues(_.map(_._2).sum)
      .toStream
}
val data = loadData("path/to/data.txt")
val processed = processData(data)

processed.take(10).foreach(println)
\end{lstlisting}
\small
\textit{Note.} In this example, the \textbf{loadData} function lazily loads data from a file and returns a stream of lines. The \textbf{processData} function defines a series of lazy transformations on the input stream, including splitting lines into words, counting word occurrences, and grouping by word. The \textbf{take} operation triggers the evaluation of the first 10 elements, and the \textbf{foreach} operation prints the results. This modular and composable approach allows for the creation of reusable data processing components that can be easily combined and adapted to different datasets and requirements.
\textit{Creator.} Author's own work.
\end{table}